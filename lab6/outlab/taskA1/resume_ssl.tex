\documentclass{resume_ssl}

\begin{document}

\begin{Header}{Akash Trehan}{Computer Science and Engineering}{II year B.Tech}{Room No. 271, Hostel 7, IIT Bombay}{Mumbai, India - 400076}{akash.trehan123@gmail.com}{photo.jpg}
\end{Header}

\rule{\textwidth}{1pt}

\begin{Section}{Academic Achievements}
\begin{itemize}

\cvitem{April 2016}{Ranked \textbf{2\textsuperscript{nd}} in the department among 126 students on CPI basis.}

\cvitem{April 2016}{Scored \textbf{AP grade} in Biology and Engineering Drawing for Advanced Performance}

\cvitem{July 2015}{Secured \textbf{99.99 percentile} in JEE Main amongst \textbf{1.5 million} students across India}

\cvitem{June 2015}{Secured \textbf{All India Rank 24}, State Rank 1 in \textbf{IIT JEE Advanced} out of 150,000 students in India}

\cvitem{June 2015}{School Topper in class 12 scoring 95\% in CBSE}

\cvitem{May 2013}{School Topper in class 10 scoring 97\% in ICSE}

\cvitem{}{Won the \textbf{National Talent Search Examination '12}(NTSE) scholarship and \textbf{Kishore Vaigyanik Protsahan Yojana '14} (KYPY) Fellowship Award.}

\end{itemize}

\rule{\textwidth}{1pt}

\end{Section}

\begin{Section}{Projects}
\begin{Subsection}{EVENTual}{Jan 2016--Present}{Institute Technical Summer Project}{IIT Bombay}{
\item {Created a platform for \textbf{creating and sharing events} easily, automating the task of filling details in the calendar by creating sharable links and QR Codes.}
\item {Implemented a \textbf{Django backend} to maintain a database, handle queries and serve the
\href{http://www.eventual.co.in}{EVENTual website.}}
\item {Developed an \textbf{Android application} with functionality including communicating with the server and QR code scanning.}
\item {Made a \href{http://www.eventual.co.in}{website platform} using \textit{HTML, CSS, JS and jQuery}.}
}
\end{Subsection}

\begin{Subsection}{MooDLD}{Sept 2015--Present}{Institute Academic Council}{IIT Bombay}{
\item {Made a \textbf{one-click Moodle downloader} in Python using Tkinter GUI for auto-downloading course files on Moodle. Saves the user for downloading all files one by one or even caring about new files being uploaded.}
\item {\textbf{Designed and tested} the corresponding Android application developed.}
}
\end{Subsection}

\begin{Subsection}{Institute Hacker News}{April 2016--Present}{Student Technical Activity Body(STAB)}{IIT Bombay}{
\item {Made a website interface to help students \textbf{share interesting technical information} they find on the internet and have discussions about them.}
\item {\textbf{Implemented the backend} using Django for rendering the website, managing database, authenticating users using Single Sign On(SSO) login, managing upvotes and comments on various shared links.}
}
\end{Subsection}

\begin{Subsection}{Checkers with AI}{Sept 2015--Oct 2015}{Guide: Prof. Varsha Apte}{IIT Bombay}{
\item {One player checkers with \textbf{Artificial Intelligence}(Computer) as opponent and using Prof. Abhiram Ranade's 'simplecpp' library for graphics.}
\item {Inspired from the \textbf{Minimax algorithm for Game AI}}
}
\end{Subsection}

\end{Section}

\rule{\textwidth}{1pt}

\quote{Bande Da Kamm Hai Bandagi Karna...Phal Dena Malak Di Mauj Hai}{Baba Nand Singh Ji}

\rule{\textwidth}{1pt}

\begin{Section}{Technical Achievements}
\normalsize
{
\begin{itemize}
\item{\textbf{Runner Up} in \textbf{Kandy Sugar Hackathon}(open to all) 2016.}
\item{\textbf{1\textsuperscript{st}} position in \textbf{XLR8 2015} - Created a Remote controlled obstacle crossing robot.}
\item{\textbf{25\textsuperscript{th}} position(participating alone) against 430 teams in \textbf{Backdoor CTF 2016}.}
\item{Succesful submission at \textbf{Microsoft code.fun.do} 2016 hackathon - Created a game using C\#.}
\item{Successful submission at \textbf{Lenovo Game Jam} 2015 - Created a game using python.}
\item{Working on \textbf{Network and Information Security} under Seasons of Code initiative  learning new hacks and participating in various Capture the Flag competitions.}
\item{\textbf{Conducted a 4-hour hands-on workshop} on \textbf{Arduino programming} for \textbf{50+} students.}
\item{Made a \textbf{Remote-controlled Airplane} and an \textbf{Autonomous Line Follower}.}
\item{Have \textbf{various personal projects} on \href{https://github.com/CodeMaxx}{my github} such as a python script for downloading all videos of your favourite youtube channel and many games made in python(with Django), \textsc{java} and C\#}
\end{itemize}
}
\end{Section}


% \begin{Course}
% \core{CS215; CS213}
% \noncore{EE101}
% \end{Course}

\end{document}