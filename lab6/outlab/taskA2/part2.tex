\section{Shortest path problem}
\subsection{Introduction}
In a graph, the `shortest path problem' is the problem of finding the minimum distance (represented by weights of edges) between two nodes of the graph.
The problem of finding the shortest path between two intersections on a road map (the graph's vertices correspond to intersections and the edges correspond to road segments, each weighted by the length of its road segment) may be modelled by a special case of the shortest path problem in graphs.
\subsection{Definition}
The shortest path problem can be defined for graphs whether undirected, directed, or mixed. It is defined here for undirected graphs; for directed graphs the definition of path requires that consecutive vertices be connected by an appropriate directed edge.\cite{dijkstra}
\subsection{Algorithms}
There are many algorithms to solve the Shortest Path problem. Some of them are-
\begin{enumerate}
	\item{\textbf{Dijkstra's algorithm} solves the single-source shortest path problem.}
	\item{\textbf{Bellman–Ford algorithm} solves the single-source problem if edge weights may be negative.}
	\item{\textbf{A* search algorithm} solves for single pair shortest path using heuristics to try to speed up the search.}
	\item{\textbf{Floyd–Warshall algorithm} solves all pairs shortest paths.}
	\item{\textbf{Johnson's algorithm} solves all pairs shortest paths, and may be faster than Floyd–Warshall on sparse graphs.}
	\item{\textbf{Viterbi algorithm} solves the shortest stochastic path problem with an additional probabilistic weight on each node.}
\end{enumerate}